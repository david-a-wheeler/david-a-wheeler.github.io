% Use only LaTeX2e, calling the article.cls class and 12-point type.

\documentclass[12pt]{article}

% Users of the {thebibliography} environment or BibTeX should use the
% gmucite.sty package.
% This package should properly format in-text
% reference calls and reference-list numbers.

%\usepackage{gmucite}

% Use times if you have the font installed; otherwise, comment out the
% following line.

\usepackage{times}

% The following packages were added to assist with formatting

\usepackage{lipsum}
\usepackage{listings}
\usepackage{xcolor}


% The preamble here sets up a lot of new/revised commands and
% environments.  It's annoying, but please do *not* try to strip these
% out into a separate .sty file (which could lead to the loss of some
% information when we convert the file to other formats).  Instead, keep
% them in the preamble of your main LaTeX source file.


% The following parameters seem to provide a reasonable page setup.

\topmargin 0.0cm
\oddsidemargin 0.2cm
\textwidth 16cm 
\textheight 21cm
\footskip 1.0cm


%The next commands set up an environment for the abstract to your paper, quotations, and coding.

\newenvironment{gmuabstract}{%
\begin{quote} \bf}
{\end{quote}}


\newenvironment{gmuquotation}{%
\begin{quote} \it}
{\end{quote}}

\lstset{
                breaklines=true,
                numbersep=5pt,
                xleftmargin=.5in,
                xrightmargin=.25in,
                belowskip=20pt
} 


% If your reference list includes text notes as well as references,
% include the following line; otherwise, comment it out.

\renewcommand\refname{References and Notes}

% The following lines set up an environment for the last note in the
% reference list, which commonly includes acknowledgments of funding,
% help, etc.  It's intended for users of BibTeX or the {thebibliography}
% environment.  Users who are hand-coding their references at the end
% using a list environment such as {enumerate} can simply add another
% item at the end, and it will be numbered automatically.

\newcounter{lastnote}
\newenvironment{gmulastnote}{%
\setcounter{lastnote}{\value{enumiv}}%
\addtocounter{lastnote}{+1}%
\begin{list}%
{\arabic{lastnote}.}
{\setlength{\leftmargin}{.22in}}
{\setlength{\labelsep}{.5em}}}
{\end{list}}


% Include your paper's title here

\title{\color{blue} Title of your paper} 

% Place the author information here.  Please hand-code the contact
% information and notecalls; do *not* use \footnote commands.  Let the
% author contact information appear immediately below the author names
% as shown.  We would also prefer that you don't change the type-size
% settings shown here.

\author
{Your name\\
\textit{Date of this version (use YYYY-MM-DD, the ISO standard for dates)}}

% Include the date command, but leave its argument blank for formatting.

\date{}



%%%%%%%%%%%%%%%%% END OF PREAMBLE %%%%%%%%%%%%%%%%



\begin{document} 

% Double-space the manuscript.
%\baselineskip24pt

% Make the title.

\maketitle 



% Place your abstract within the special {sciabstract} environment.

\begin{gmuabstract}
  Type your abstract here (note this falls within the \textit{gmuabstract} boundary.  \lipsum[1]
\end{gmuabstract}



% The \section command numbers corresponding sections for the paper.
% Use the \subsection and \subsubsection command for managing
% subsections within your paper.

\section{Introduction}

As per the requirements, this template provides a method to set the "paragraph type" of each paragraph correctly.  You should \textit{never} touch the font name, font size, or similar properties of ordinary paragraphs if your paper is longer than a page (aka "direct formatting"); you should \textit{only} set those kinds of properties of paragraph types.  This template provides several paragraph types, including: 

\begin{itemize}
    \item "Body Text" (Latex) for normal text 
    \item "{\textbackslash}section" for level-1 headings 
    \item "{\textbackslash}subsection" for level-2 headings 
    \item "thebibliography" for endnote bibliography entries 
    \item "lstlisting" (or similar) for code 
    \item "gmuquotation" for quotations 
\end{itemize}

\subsection{More stuff}

You might use numbered lists:

\begin{enumerate}
    \item \lipsum[1] 
    \item \lipsum[2] 
\end{enumerate}

You might want a long quote; this uses the paragraph style \textit{gmuquotation}: 

\begin{gmuquotation}
    Four score and seven years ago our fathers brought forth on this continent, a new nation, conceived in Liberty, and dedicated to the proposition that all men are created equal. \cite{Lincoln:1863}
\end{gmuquotation}

Or code (this uses the \textit{lstlisting} paragraph type, just drop your code in): \\

\begin{lstlisting}
#include <stdio.h>

int main() {} 

\end{lstlisting} 

You can switch back to "Body Text" when done. \\


\section{Another top-level heading}

Use author-date format for citations, e.g., there are many ways that Heartbleed could have been countered \cite{Wheeler:2014}.  Insert these citations into the thebibliography section found at the end of this text file.  Formatted examples are included.  If citations appear as "[?]" after running PDFLaTeX once, run PDFLaTeX a second time to fix the citation.\\

Add new paragraph types if needed.  Always include a "{\textbackslash}{\textbackslash}" at the end of each paragraph to ensure spacing is correct. \\

\lipsum[1]


% Your references go at the end of the main text, and before the
% figures.  For this document we've used BibTeX, the .bib file
% gmubib.bib, and the .bst file GMUpaper.bst.  The package bibcite.sty
% was included to format the reference numbers according to GMU
% style.

\section{References}



%----------------------------------------------------------------------------------------
%	BIBLIOGRAPHY
%----------------------------------------------------------------------------------------

\begin{thebibliography}{99} % Bibliography - this is intentionally simple in this template

%\bibitem[Figueredo and Wolf, 2009]{Figueredo:2009dg}
%Figueredo, A.~J. and Wolf, P. S.~A. (2009).
%\newblock Assortative pairing and life history strategy - a cross-cultural
%  study.
%\newblock {\em Human Nature}, 20:317--330.


\bibitem[Lincoln1863]{Lincoln:1863}
Lincoln, Abraham. 
\newblock {\em The Gettysburg Address}. 1863.

\bibitem[Wheeler2014]{Wheeler:2014}
Wheeler, David A..
\newblock {\em How to Prevent the next Heartbleed}. 2014-10-20.
\newblock URL: https://www.dwheeler.com/essays/heartbleed.html.
\newblock {\em Accessed 10/31/15}

%\bibitem[OWASP2009]{OWASP:2009}
%The Open Web Application Security Project.
%\newblock {\em Heap overflow}. 2009.
%\newblock URL: https://www.owasp.org/index.php/Heap\_overflow.
%\newblock {\em Accessed 9/22/15}

%\bibitem[Conover1999]{Conover:1999}
%Conover, Matt (A.K.A. Shok) and w00w00 Security Team.
%\newblock {\em w00w00 on Heap Overflows}. 1999.
%\newblock URL: http://www.cgsecurity.org/exploit/heaptut.txt.
%\newblock {\em Accessed 9/22/15}

%\bibitem[Prinz2005]{Prinz:2005}
%Prinz, Peter and Crawford, Tony. 
%\newblock {\em C in a Nutshell}. 2005.
%\newblock  O'Reilly Media, Inc. ISBN-10 0596006977.

%\bibitem[MITRE2014]{MITRE:2014}
%MITRE.
%\newblock {\em CWE-122: Heap-based Buffer Overflow}. 2014.
%\newblock URL: http://www.cgsecurity.org/exploit/heaptut.txt.
%\newblock {\em Accessed 9/22/15}


\end{thebibliography}

\clearpage

\end{document}




















