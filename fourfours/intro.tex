%  Four fours
%      Copyright (C) 2001-2002 David A. Wheeler
%      dwheeler@dwheeler.com 
%  Original version of this paper 11 September 2001.

\documentclass[twocolumn,twoside,draft,american]{report}
\usepackage[american]{babel}
\selectlanguage{american}
\title{The Definitive Four Fours Answer Key}
\author{David A. Wheeler}
\date{14 June 2002}
\begin{document}

\maketitle

%  Begin the document 

\chapter{Introduction}

This is the definitive answer key for the ``four fours'' problem.
The goal of the four fours problem is to find a mathematical expression
for every whole number from 0 to some maximum, using only common mathematical
symbols and exactly four fours (no other digits are allowed).
I call this the ``definitive'' answer key because, at the time of this
writing, this key has more answers than any other source in the world.
For more information about this answer key (or a copy of it),
see http://www.dwheeler.com/fourfours.

It turns out there are many variations of the ``four fours'' problem.
One kind of variation allows fewer than four fours, and prefers fewer fours;
I'm not describing that variation here.
The many other variations differ by which mathematical operations are allowed.
Rather than create a single set,
I've devised an ``impurity'' level for any expression.
Expressions with the lowest impurity are considered ``better'' than any
expression with a higher impurity.  Here are the impurity levels, along with
the operations allowed at that level:

\begin{description}
\item[0:]
 The operations addition (+), subtraction/negation (-),
  multiplication ($\cdot$), division (/),
  square root ($\sqrt{x}$), factorial (!), and power ($x^y$).
 Parentheses may be used for grouping.
 The digit 4 must be used exactly four times, and the decimal digit (.)
 can be used.
\item[1:]
 This level isn't used.  Originally I put square root and powers in here,
 but later I decided that they belonged with level 0.
 Some numbers don't need to be represented with these advanced functions,
 but I prefer their representation that way.
\item[2:]
 The overline, an infinitely repeated digit.
   For example, $.\overline{4}$ is 4/9.
\item[3:]
 An arbitrary root power.  For example, ${\sqrt[.4]{4}}$ is 32.
\item[4:]
 The gamma function. For example, $\Gamma(4)$ is 6;
 in general $\Gamma(x) = (x-1)!$.
\item[5:]
 The percent symbol, \%.  $4\%$ is 0.04.
\item[6:]
 The square function.  $sq(4)$ is 16.
\item[7:]
 The logical-or ($\lor$), exclusive-or ($\oplus$), and logical-and ($\land$).
 These operators use the binary representations of their left and right
 sides, and compare the corresponding binary digits of each ``input'' number.
 In logical-or, the result is 1 if either input is 1 (else 0).
 In exclusive-or, the result is 1 if the inputs differ (else 0).
 In logical-and, the result is 1 if both inputs are 1 (else 0).
 Some examples may help.  Since 12 has the binary value $1100$, and
 10 has the binary value $1010$, we can compute the following:
 $12 \lor 10 = 14$ (binary $1110$),
 $12 \oplus 10 = 6$ (binary $0110$), and
 $12 \land 10 = 8$ (binary $1000$).
\item[8:]
 The logical left shift ($<<$)  and right shift ($>>$).
 These operators take the binary representation of the left-hand number,
 and shift those bits by the number of positions specified by the
 right-hand number.
 When shifting to the right, the rightmost bits ``disappear''; when
 when shifing to the left, the rightmost bits being added are
 set to zero.
 Since these are primarily computer operations, and are arguably less common,
 I've assigned them the worst impurity.
\end{description}

There are many operations I'm intentionally excluding.
I don't include the ``logical not'' operator, because it only makes
sense given a finite maximum number of bits (and picking any
particular size would be too arbitrary).
I don't include the trigonometric functions - some people use trigonometric
functions in degrees, but the usual definition of these functions
use radians (which are not very useful for the problem).
I don't include operations such as ``round'', ``floor'', and
``ceiling'' - these are approximation operators and don't seem appropriate
(many other people seem to feel the same way).

I don't include the ``log'' operators, and for an interesting reason;
there's a way to use any log operator to create any number.
Here's the explanation from the ``Four Fours Problem'', a compilation
by Paul Bourke.
% http://astronomy.swin.edu.au/~pbourke/fun/4444/
He credits Ben Rudiak-Gould with this description of how natural
logarithms (ln()) can be used to represent any positive integer n as:

$n = -ln[ ln( sqrt(sqrt(...(sqrt(4))...))) / ln(4) ] / ln(4)$
where the number of nested sqrt() functions is twice n.

There are other operators that I haven't considered at this time.
This includes the greatest common denominator gcd(x,y),
least common multiple lcm(x,y), the ``mod'' (modulo) function,
and the Euler function $\Phi(x)$ (which counts the numbers between
1 and x whose gcd with x is 1).
Functions for permutations and combinations could be added, too.
However, no one else seems to be using those operations for the four fours
problem, so I decided to not include them (at least at this time).

It could be argued that I should prefer the ``percent'' operator (\%)
over the ``gamma'' operator (i.e., that their impurity levels should be
swapped).
I preferred the gamma operator because it seemed
to me that the gamma operator was cleaner mathematically.
I've sometimes wondered if that's really true, and perhaps
someday I'll change my mind.
At the least, it's a debatable choice.

For many numbers, there's more than one way to represent the number;
I always choose the solution with the
smallest impurity, then the fewest number of operations with that impurity.
For example, $(4/4)^{4/4}$ is another way to compute 1, but it
requires 3 operations (two divisions and one power), and since $44/44$
only requires one operation (one division), I'll use $44/44$ instead.
For this purpose, I counted percent and the overline as ``operators'',
and I don't count parentheses as operators.
Because I prefer solutions that have the fewest number of operations, some
of the results shown here are quite unusual.
For example, most people would represent 16 as $4+4+4+4$; this is valid,
but the expression $.4*(44-4)$ only requires 2 operations instead of 3
so I'll use that instead.

I'm using the ``usual'' notation for these operations.
I used a center-dot to emphasize multiplication, and I used ``/'' to
represent division so that less space will be used when showing all these
numbers.
Factorials are done before anything else,
then powers, then multiplication and division, then
addition and subtraction,
then left and right shift, then logical and,
then exclusive or, then logical or.
Parentheses override this order, and the overline (e.g., $.\overline{4}$)
and percent sign are considered part of the number.
I usually don't parenthesize where it's not necessary.
For example, $4/.4/.4$ has the value 25.

Some people who have tried the four fours problem may still be
surprised by some of the answers given here, because some of them
use some unusual mathematical expressions.
For example, using 2 fours, you can represent the numbers one through 12:
1 ($4/4$),
2 ($4-{\sqrt{4}}$),
3 (${\sqrt{4/.\overline{4}}}$),
4 (${\sqrt{4 \cdot 4}}$),
5 (${\sqrt{4}}/.4$),
6 (${4!}/4$)
7 (${\Gamma({\sqrt{4}})}+{\Gamma(4)}$)
8 ($4+4$)
9 ($4/.\overline{4}$)
10 ($4/.4$)
11 (${\sqrt{{\Gamma({\sqrt{4}})}+{\Gamma({\Gamma(4)})}}}$)
and 12 (${4!}/{\sqrt{4}}$).
My favorite with two fours is 32, which can be represented as
${\sqrt[.4]{4}}$.
By starting with these building blocks, you can represent lots of numbers.

Note that there's no need to list negative numbers.
Any value $-x$ could be
formed by finding the positive $x$ and then changing the expression
to $-(x)$.

There are other collected answers for the Four fours problem.
That includes the
comp-sci collection at http://www.comp-sci.demon.co.uk/FourFoursSo.html,
Paul Bourke's collection  (with Frank Mrazik) at http://astronomy.swin.edu.au/~pbourke/fun/4444 (but note that some solutions use non-standard notation!),
the collection of ``interesting'' solutions at http://www.wheels.org/math/44s.html,
and
the Math Forum/Ruth Carter's list at http://mathforum.com/ruth/four4s.puzzle.html.
Pete Karsanow's Four Fours FAQ at
http://www.geocities.com/TimesSquare/ Arcade/7810/44sfaq.htm
emphasizes solutions based on the book for Texas Instruments
(TI) calculators, which is where I first learned of this problem too
(note: this site has download limitations and sometimes isn't available;
use the Internet Archive to load old versions if necessary).
A Google search of ``four fours'' will find many interesting sites.

The four fours problem isn't really a deep mathematical problem, since
it completely depends on the oddities of common mathematical notation.
See Mathnet's discussion of this at
http://www.math.toronto.edu/mathnet/ questionCorner/fourfours.html.
Still, it's fun!

The list of answers is formatted as the value,
the impurity level in parentheses, an equal sign, and the expression
(using exactly four fours) that equals that value.
In a few places, the PDF version has formatting problems;
if you want to see the exact equation, I've also posted a text
version of the solutions.
Currently, I list whole numbers from 0 up to 40,000.
The first missing entries (for which I know no solutions at all) are
2179,
2227,
2263,
2467,
and
2611.
The first numbers which use a higher impurity level than any
of its predecessors are 73, 113, 197, 1651, and 2237.

Enjoy!


\chapter{Answers}

